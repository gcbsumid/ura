\begin{quote}
        {\sl
        ``I think this assignment should have 8 objectives but we should still mark it out of~10.''
        } \\
        \mbox{}\hfill -- The Mean TA.
\end{quote}

This assignment is due {\bf \AoneDeadline}.

\section{Topics}

\begin{itemize}
	\item Exposure to OpenGL
	\item Callback-based program structure
	\item \texttt{Qt} user interfaces
\end{itemize}

\section{Statement}

This assignment will get you started writing graphics applications
using OpenGL.  It will also familiarize you with the set of languages
and APIs we will be using for subsequent assignments.  You will be
writing a user interface in C++ using the \texttt{Qt} toolkit.
The user interface will be wrapped around a window in which graphics
will be rendered using OpenGL from the C++ application.

In particular, the program is a game in which the user must manipulate
falling tetrominoes so as to make complete lines at the bottom of a
well.  When lines are completed, they are removed from the game.  When
the well fills up, the game is over.  Any resemblance between this
game and a popular arcade game of Russian extraction from the 1980s
is purely coincidental.

You will also implement some graphical functionality not directly
related to game play, but that will help you develop Modern OpenGL skills
needed in later assignments.

\section{The game}

The game takes place in a U-shaped well of unit cubes enclosing
a grid of width 10 and height 20 in which tetrominoes can fall.
The blocks occupy discrete positions in the grid (they don't fall
smoothly, but jump from position to position).

Tetrominoes start in a four unit tall stripe on top of the well.
Every time a predetermined interval elapses, the current tetromino
falls one unit.  A value of 500ms is a good novice interval; 100ms is more
challenging.  At any time, the current piece can be moved to
the left or right, rotated clockwise or counter-clockwise, and
dropped the rest of the way down the well.  When the piece can
fall no further, it stops, and any rows in the well that are completely
filled are removed from the game.  The game ends when a piece
cannot clear the starting stripe.

You should have at least three different speeds at which the pieces fall.

The game should be drawn from unit cubes.  The well and the
various piece shapes should all be drawn in different colours.

You should add a new game piece.  The piece must not be a rotation or
reflection or shift of an existing piece.  However, your new piece
does not need to be a tetrominoe; i.e., it may be composed of more (or less)
than four cubes.

\section{The interface}

The user interface should be written in \texttt{Qt}, a cross-platform
application and UI framework for C++ developers. You will need to
implement the following functionality (the letters in () indicate the
keyboard shortcut; remember, both upper and lower case should work for
the keyboard shortcut):

\begin{itemize}
	\item An \texttt{Application} menu with the following menu items:
	\begin{itemize}
		\item \texttt{New game (N)}: Start a new game.
			\texttt{N}.
		\item \texttt{Reset (R)}: Reset the view of the game.
		\item \texttt{Quit (Q)}: Exit the program.  
			(This one should already
			be implemented; be sure not to break it.)
	\end{itemize}

	\item A \texttt{Draw Mode} menu with the following menu items:
	\begin{itemize}
		\item \texttt{Wire-frame (W)}: Draw the game in wire-frame mode.
		\item \texttt{Face (F)}: Fill in the faces in the game.  Each
			different piece shape should have its own uniform colour.
		\item \texttt{Multicoloured (M)}: Similar to \texttt{Face} mode,
			but each cube has six faces of different colours
                        (i.e., no two faces should have the same colour).
	\end{itemize}

	The \texttt{Draw Mode} menu should use radio buttons to
	indicate which state is selected.

	\item A radiobutton \texttt{Speed} menu with at \texttt{Slow (1)}, 
          \texttt{Medium (2)}, and \texttt{Fast (3)} 
          speeds that sets the rate at which pieces fall.  
		The game may use additional speeds, but you need
		to be able to set the speed to one of the three.

\iffalse
	\item A \texttt{Buffering (B)} menu with the following menu item:
	\begin{itemize}
		\item \texttt{Double buffer}: Toggle double-buffering. This should be a check item.
	\end{itemize}

	The keyboard shortcut \texttt{B} should toggle the buffering mode.
\fi

	\item Mouse movements:
	\begin{itemize}
		\item Mouse operations should be initiated by pressing the
		  appropriate mouse button and terminated by releasing
		  that button.  Only motion in the horizontal direction
		  should be used.
		\item The left mouse button should rotate the game around the
		  X axis.
		\item The middle mouse button should rotate the game around
		  the Y axis.
		\item The right mouse button should rotate the game around
		  the Z axis.
		\item When the shift key is pressed, all mouse buttons should
		  uniformly scale the game (both the board and the pieces).
		  When the mouse moves to the left, the game should become
		  smaller, and when the mouse moves
		  to the right the game should become larger.  
		  The maximum and minimum scales should be restricted to
		  a reasonable range.
	\end{itemize}

	You must make reasonable decisions about how much to scale
	or rotate for every pixel's worth of mouse motion.   For example,
	if the mouse isn't moving, there should be no scaling or rotation.

	You are also required to implement a feature sometimes known
	as ``persistence'' or ``gravity''.  If, while rotating, the mouse
	is moving at the time that the button is released, the rotation
	should continue on its own.  This decision should be made at the
	time of release; after that, it should persist independently of
	mouse movement, until the next button press.

	\item Game play:
	\begin{itemize}
		\item The left arrow key should move the currently falling piece
		  one space to the left.
		\item The right arrow key should move the current piece
		  one space to the right.
		\item The up arrow key should rotate the current piece
		  counter-clockwise.
		\item The down arrow key should rotate the current piece clockwise.
		\item The space bar should 'drop' the piece, sending it as far
		  down in the well as it will go.
	\end{itemize}
\end{itemize}

\section{Qt and OpenGL}

In this course, user interfaces are written in \texttt{Qt}, a cross-platform
application and UI framework for C++ developers. 

If you have opt to not to use qt, you will need to use the following OpenGL
commands: \\

\texttt{Shader Program}
\begin{itemize}
    \item glCreateShader
    \item glSourceShader
    \item glCompileShader
    \item glDeleteShader
    \item glCreateProgram
    \item glAttachShader
    \item glLinkProgram
    \item glDetachShader
    \item glDeleteProgram
    \item glValidateProgram
    \item glUseProgram
    \item glGetAttribLocation
    \item glGetUniformLocation
\end{itemize} 

\texttt{Drawing Objects}
\begin{itemize}
    \item glGenVertexArrays
    \item glBindVertexArray
    \item glGenBuffers
    \item glBindBuffer
    \item glBufferData
    \item glEnableVertexArray
    \item glVertexAttribPointer
    \item glDrawArrays
    \item glDisableVertexArray
    \item glVertexAttrib*
    \item glUniform*
\end{itemize}

\texttt{Other}
\begin{itemize}
    \item glEnable
    \item glDisable
    \item glClear
    \item glClearColor
\end{itemize}

(Of course, you may find that you will want to use additional OpenGL
functionality to add extra features.
Note that the \texttt{Matrix4x4} class stores its matrices in row-major
order, but OpenGL expects matrices in column-major order.)

If you have chosen to use qt, you will use the following objects which replaces
some OpenGL calls and classes you have to make.

For more information, go to \href{http://qt-project.org/doc/qt-5/index.html}{Qt
Documentation}. \\

\texttt{Classes}
\begin{itemize}
    \item QOpenGLBuffer (replaces gl(Gen|Bind)Buffer(data)) - Qt 5.1+
    \item QOpenGLArrayBuffer (replaces gl(Gen|Bind)VertexArray) - Qt 5.1+
    \item QGLBuffer (same as QOpenGLBuffer) - Qt 5.0
    \item QGLShaderProgram (replaces all Shader Program calls)
\end{itemize}

Each class has replaces certain OpenGL calls while making the interface simpler.
Please look at the documentation for the full class descriptions. \\

\iffalse
\subsection{Double buffering}

Your program should support both single and double buffering.  In single
buffering, graphics commands draw directly onto the screen.  In
double buffering, drawing happens on a ``back buffer'' that is then
swapped with the ``front buffer'' when drawing is complete. Because
the swapping happens quickly, the user never sees a half-complete rendering.

The OpenGL command \texttt{glDrawBuffer} lets you choose which buffer
to draw into.  The \texttt{QGLWidget} allows you to swap the
buffers, and this is already implemented for you in \texttt{viewer.cpp}
\fi

\section{The Skeleton Program}

If your account is correctly set up, you will find a skeleton
program in the \texttt{cs488/A1/src} subdirectory of the source
distribution.  The program creates a user interface
with an OpenGL window.   As a test, it draws triangles where the
corners of your game (not including the well) should appear.
The camera is set up so that the triangles appear centered and
correctly sized.
You need to modify this code to
render the current state of the game and respond to user interface
events.  Here's a to-do list, with a suggested order that will help
you make your way through the assignment.

\begin{itemize}
	\item Write a function to draw a unit cube using OpenGL. 

            If you're already familiar with Modern OpenGL, you can
            write a single cube to a Vertex Buffer Object (VBO). You
            would then create a \texttt{Matrix4x4} model matrix for
            each cube and use the matrix functions to translate,
            rotate, scale this model. 

	\item In your render function, draw a U-shaped border for
	      the well out of cubes.

	\item Implement face rendering and wireframe
		  rendering.

	\item Implement rotation and scaling.  You should be
	      able to see the effect on the well.

\iffalse
	\item Implement single- and double-buffering.
		  You should be able to see the difference when rotating
		  the well.
\fi

	\item A new piece shape has been added.

	\item Add code to draw the current contents of the game.
		  Each piece type should be drawn in a different colour;
		  the colours are up to you.

              Note that color manipulation happens in the fragment shader.

	\item Hook up a simple timer (using the \texttt{QTimer}
		  class) that calls down to the game's
		  \texttt{tick} method and re-renders.  You should be able to see
		  pieces falling.

	\item Implement the rest of the controls for game play and
		  the remaining user interface details.
\end{itemize}

\section{Donated code}

The skeleton program comes with the following files:

\begin{itemize}
        \item \texttt{main.cpp} -- The entry point for the program.
        \item \texttt{viewer.hpp}, \texttt{viewer.cpp} -- The OpenGL
          widget. All of the OpenGL-related code is here.
        \item \texttt{appwindow.hpp}, \texttt{appwindow.cpp} -- The
          application window code. Most of the UI-related code
          (menubars, etc.) is in these two files.
	\item \texttt{game.cpp}, \texttt{game.hpp} -- An engine that implements
		the core of the falling blocks game.
        \item \texttt{game488.pro} -- Used to create a Makefile that includes
            all qt libraries
	\item \texttt{Makefile} -- Used to compile the program with \texttt{make}.
        \item \texttt{shader.vert} -- Vertex Shader used to determine the
            position of every vertex
        \item \texttt{shader.frag} -- Fragment Shader used to determine the
            color of each pixel
\end{itemize}

You should be able to get the skeleton program running using the
commands \texttt{qmake game488.pro; make; ./game488}.

\section{Deliverables}

These executable files should be put in the directory \texttt{cs488/handin/A1}:

\begin{itemize}
	\item \texttt{game488} -- The program executable.
\end{itemize}

All source files should be in the directory \texttt{cs488/handin/A1/src}.

Don't forget \texttt{screenshot01.png}

\section{Hints, tips, and ideas}

There are lots of ways this simple application could be modified
to enhance playability and attractiveness.  You are encouraged to
experiment with the code to implement these sorts of changes, as
long as you have already met the assignment's basic objectives.
Here are some suggestions:

\begin{itemize}
  \item As the game progresses (ie, as more filled rows are removed from
    the game), have the pieces fall at a faster rate.  

	\item A scoring mechanism.
	\item Head-to-head networked play.
	\item Modified cubes for pieces.  The blocks look much better
		  if individual cubes have their edges slightly beveled.
	\item Animations for certain events.  The board can spin around
		  when you lose, for example.
	\item Add lighting.
\end{itemize}

If you make extensive modifications to the game, you should make
sure to run in a ``compatibility mode'' mode by default -- you should
support at least the user interface required by the assignment.
You can activate your extensions either with a special command line
argument or a menu item.

\newpage
\section{Objectives: \hfill Assignment 1}
\ \\ {\large \bf Due: \AoneDeadline.} \\
\ \\ {\large \bf Name: \hrulefill} \\
\ \\ {\large \bf UserID: \hrulefill} \\
\ \\ {\large \bf Student ID: \hrulefill} \\
\begin{description}
        \item[\_\_\_ 1:]
          Wireframe mode works.
        \item[\_\_\_ 2:]
          Face colour mode works.
        \item[\_\_\_ 3:]
       	  Multicoloured face mode works.
        \item[\_\_\_ 4:]
          Pieces fall at three or more speeds.
        \item[\_\_\_ 5:]
          A new piece has been added to the game.
        \item[\_\_\_ 6:]
          The user interface works as specified (menus, mouse
          interaction, etc).
        \item[\_\_\_ 7:]
          The game can be rotated.
        \item[\_\_\_ 8:]
          The game can be scaled.
        \item[\_\_\_ 9:]
          The game is playable (i.e., you can move the pieces as described 
		under "game play" of the assignment specification).
        \item[\_\_\_ 10:]
          Persistence works for rotation.
\end{description}

\vspace*{.1in}
{\bf Declaration:}
\begin{quote}
        I have read the statements regarding cheating in the \CourseNumber
        course handouts.
        I affirm with my signature that I have worked out my own solution
        to this assignment, and the code I am handing in is my own.

\end{quote}

        {\bf Signature:}
\vspace*{.1in}


%%% Local Variables: 
%%% mode: latex
%%% TeX-master: "a1.tex"
%%% End: 
