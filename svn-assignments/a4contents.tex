%\Chapter{Assignment 4}
\begin{quote}
	{\sl 
	``If you want to do a ray tracer for your project, you 
	\/{\bf really} need
	 to complete Assignment 4 first.''
        }\\
\end{quote}
This assignment is due {\bf \AfourDeadline}.

If you still need the provided code for this assignment run
 \texttt{\CourseData/bin/setup A4} from your CSCF account.

\section{Topics}
\begin{itemize}
    \item Ray tracing.
    \item Inverse transformations and ray transformation.
    \item Ray intersection with spheres, cubes, 
	  and meshes of convex planar polygons.
    \item The Phong lighting model.
\end{itemize}

\section{Statement}
A ray tracer comprises a few major pieces of code to 
\begin{enumerate}
    \item Cast rays from the eye point through each pixel,
    \item Intersect the rays with the objects in the scene, 
    \item Cast shadow rays from the point of intersection to each light source, and
    \item Shade the rays, summing the contribution from each visible 
	  light source, and assign a colour to the pixel.
\end{enumerate}
You are to write such a ray tracer.
This ray tracer needs to support spheres, cubes, 
and meshes of convex, planar polygons as its basic geometric primitives.  
The ray tracer modeling language will be implemented
as a set of Lua callbacks, so scenes---including procedurally
generated scenes---are specified as Lua scripts.

If you are planning to render a large scene (as opposed to
a small test scene, small being less than 32x32 rays) please 
use the machines in {:LabRoom:}: raytracing
takes a lot of sustained CPU time, which CFCF frowns upon for 
time-shared machines.   On shared CFCF machines, if 
you render small test images, please ``nice'' your rendering jobs.
Note also that many shells put a limit on how much CPU time a single
program can use.  To avoid running into this limit, run {\tt unlimit cputime}
before starting your ray tracer.

% The final images can be displayed on an SGI or Sun using the {\tt ImageMagik}
% tool {\tt display}.  The tool {\tt xv} can also be used to display
% images, but note that it is a copyright violation to use {\tt xv3} 
% (version 3 or higher)
% for displaying course work.  There is also an SGI viewer called
% {\tt imgview} that works on the Octanes.   All these viewers can
% display the PNG files used in this assignment, as can most current
% web browsers.

You are required to implement the following ray tracing functionality:
\begin{itemize}
	\item Hierarchical transformations and models.
	\item The sphere, cube, and (convex, planar) polygonal mesh 
              geometry types.
	\item Bounding volumes (sphere or boxes) for polygonal mesh types.
	\item Point light sources, with a Phong lighting model.
	\item Output of the resulting image to a PNG file.
	\item One additional feature of your own choosing, and 
                a scene file (and resulting image) 
		that demonstrates this feature.
\end{itemize}
{\bf Note:} you may assume that all faces of a polygonal 
meshes are convex and planar.   
It is \emph{suggested} that an
intersection-of-halfspaces approach be used in 2D to test
if an intersection point with the plane of a polygon lies within
that (convex) polygon's area.  
Project the 3D plane/ray intersection point into 2D by
dropping the coordinate corresponding to the largest value in
the polygon's plane normal.

Use face normals for the cube and polygonal mesh
model types to support shading.  
You do {\em not} have to support vertex normals.  
Note that Phong shading, i.e. normal interpolation, is
\emph{not} required---you don't have vertex normals anyway on
the polygonal mesh type provided.
However, you can implement Phong shading as your extra feature,
as long as you can still read the existing test scripts.
It is suggested that if you do this you add a new polygonal
mesh primitive with a new Lua command, rather than modifying the
interface to the one provided here.   In general, implement
the interface to your new features in a backward-compatible way.

You are also required to generate a test script named {\tt sample.lua}
that demonstrates the following:
all of the primitive types required, the 
point light sources, at least one ``shiny'' surface 
(using the specular Phong reflectance model), and your additional
features.
This script should just write
the resulting image to a {\tt sample.png} file, and exit.
You can look at the sample files in 
{\tt \CourseData/data/A4} 
to get some ideas. You can also obtain some images corresponding to
the test scripts from the course web page. These can be found
in the A4 test scene link, under the gallery/exhibition section.
When you submit the images rendered from these given scripts,
please use the ORIGINAL scripts without any modifications. If you
want to use these scripts to demonstrate your extra objective(s),
such as transparency, reflectivity, etc., be sure to include the
images generated from the unmodified scripts as part of you sample
images.

You are only required to implement primary and 
shadow rays.  Recursive secondary rays, needed to support
reflections and transparency, are {\em not\/} required.
Technical note: attenuation applies only to shadow
rays, not primary rays!

You are also required to provide an interesting background that does not
interfere with the objects in the scene.  This means that colours should
not be too bright, or the patterns too varied.  Sunset effects could 
be tried, just as a suggestion.  
A constant background is not acceptable.   You can parameterize the
background either by pixel position or ray direction.
The latter parameterization can be used, for instance, 
to make a sky background with dark blue at the zenith, 
white at the horizon, and brown below the horizon, or to implement
a starfield for an outer space scene.  Use your imagination.
Your background should be in all your raytraced pictures.

The easiest way to implement a hierarchical raytracer is to transform
each {\em ray\/} from the Viewing Coordinate System (VCS) to the
Modeling Coordinate System (MCS) of each model.  
The transformed ray is then
intersected with models in a canonical position, so you can
take advantage of special forms of equations.  
To search the entire scene, for
each ray you can do a recursive tree traversal, generating
transformation matrices as you go.    Alternatively, you can
precalcuate all VCS-to-MCS transformations, flattening the tree;
this will be somewhat faster (flattening cannot count as an extra
objective, though).
Keep in mind that in either case
light sources should be specified in WCS, so you will have to transform
the intersection from MCS back to WCS to do shading.

To compute the matrix used to transform the rays, 
you should extend the model callbacks to
maintain an {\em inverse} transformation in each model node.  
Then compose these in reverse order as you descend the tree to find the
necessary VCS to MCS transformation for each primitive.  
Object intersections then can be
made more efficient (and simpler) by using a canonical form of an
object in MCS.

\section{Suggested Development}
You are free to develop your code as you like.  However, you will
probably find it easiest if you first write a non-hierarchical
raytracer (Objectives 1-6).  Once you have that part of the code
debugged, add the hierarchical part (Objectives 8,9; affine
transformations, a general hierarchy, and bounding volumes for
polyhedral objects).  Depending on
what you implement for your extra improvement, you may or may not want
to complete Objective 10 before starting on hierarchical
transformations.

Finally, you need to make a unique scene (Objective 7).  While you
can write a unique scene earlier in the development cycle, you
may want to wait until you know if you will finish hierarchical
models, since the power of hierarchical modelling will let you
build a more interesting scene.

For this assignment, you are allowed to have some console
output.  For instance, you may want to output your render
parameters and have some indication of how much progress your
raytracer is making (i.e., 10\%, 20\% done etc).  Printing out
your entire hierarchy tree is probably too much output, though.

\section{Cautions}
This assignment is a lot of work.  Although things look easy
in the course notes, the details are a bit tricky to work out.
What follows are a few hints from students who have worked in
this assignment in the past.
\begin{itemize}
	\item Numerical problems abound.  In particular, watch
		for the following:
	\begin{itemize}
		\item Try to minimise the number of times that 
			you normalize vectors and normals.  
			Each time you normalize, you introduce
			a small amount of error that can cause
			major problems.
		\item The intersection of the ray and an object
			may actually be slightly inside the object.
			When casting secondary rays (such as shadow
			rays), the first object the secondary rays
			will hit will be the same object.  To
			avoid this problem, discard all intersections
			that occur too close to the ray origin.
		\item Use ``epsilon'' checks in your intersection
			routines, particularly the ray-intersect-polygon
			routine.
	\end{itemize}
	\item In hierarchical ray tracing, on ``the way down''
		you should transform the point and vector forming
		the ray with the {\it inverse transform.}  On
		``the way back up'' you should transform the
		intersection point by the transformation, and
		(assuming you represent the normal as a column
		vector) you should transform the normal with
		the {\it transpose of the inverse transform}.
		Potentially, this transformation of the normal
		will result in a non-zero 4th coordinate, in
		which case you should set the 4th coordinate
		to 0 (or, write a special matrix-vector product
                that ignores the fourth coordinate, and uses the
		transpose).

                To speed things up you way want to precompute and
                cache all the VCS-MCS transformations.
                Expand the DAG into a strict
                hierarchy first if you use caching.
\end{itemize}

\section{Possible Raytracer Extensions}
You are required to implement an ``additional'' non-trivial feature.
There are many possibilities, such as an
efficiency improvement or an addition of functionality.  Several ideas
are given below.
They are purely a list of suggestions for your
consideration; a list of papers on ray tracing is given in the
course bibliography.
These papers contain a lot more possibilities
(some of them might even include code.)

\subsection{Extra Functionality}
\begin{description}
\item[Mirror reflections:] This involves (recursively) issuing 
    secondary reflection rays from the point of intersection.
    This would apply to
    objects that have been assigned a surface property with a non-zero 
    ``mirror coefficient''.

    Since purely reflective objects are rare,
    blend the colour found recursively with
    the colour used for shading a semi-reflective object using shadow rays,
    using the mirror coefficient.
    Note that you will have to specify a maximum recursion depth.
    If you use mirror coefficients $r < 1$, 
    then the magnitude of $r^n$ for the maximum
    $r$ in the scene will give you an idea of how much error will
    be made when truncating the recursion in scenes with multiple
    reflective objects.   Even a recursion depth of 1 can generate
    interesting pictures, though.
\item[Refraction:] This involves generating (recursively)
    secondary refracted rays
    for objects that have a ``transparency coefficient'' and an index
    of refraction.
    Implementation is very similar to reflection rays. 
    Use Snell's law to compute the direction of the refracted ray.
	[Whitted]

    Aside: if you want to get fancy, the ratio between reflection and
    refraction is given by a function that gives more reflection
    at glancing angles.   This is called the Fresnel effect and
    is a consequence of Maxwell's Laws; see the text or the references.   
    For the purposes of
    this assignment, you can use constant coefficients for the amount of 
    reflection and refraction.    However, watch out for total
    internal reflection.
\item[Supersampling:] This involves
    generating multiple rays for each pixel and using some averaging
    function to 
    combine the colours returned (e.g. sample on a $3\times 3$
    grid over the area of the pixel and
    average the nine colour values that are returned).
    This helps to reduce the ``jaggies'' and is the most basic form
    of {\bf antialiasing}.
\item[Other antialiasing methods:] Generate more rays only where the 
    scene changes (adaptive sampling), use random (stochastic) sampling 
    techniques (jitter the sample positions in the subpixel grid), etc.  
    There are many, many antialiasing techniques.
    [Cook : Stochastic] [Dippe :  Stochastic] [Lee Uselton] [Mitchell] 
    [Painter : Adaptive-Progressive Refinement]
\item Note: If you choose to implement a supersampling objective, make
    sure you include at least two comparison images of the same scene,
    one with supersampling turned on and the other without.
\item[Fisheye/Omnimax Projection:] 
    Ray trace a 180 degree view, using a hemispherical ``screen''.
\item[Spherical Lens Systems:] Simulate a real lens system; use
    stochastic sampling across the aperture to simulate depth of field,
    and refractive intersection with sphere surfaces to create a lens system.
    (Even one lens is interesting; more would make a good project).
\item[Additional primitives:] Extend the modelling language and add primitives 
    for (truncated) cylinders and cones.   
    Note that these primitives are basically
    particular quadrics intersected with a pair of halfspaces, and (truncated)
    paraboloids and hyperbolids are in the same category.   Quadric-based
    primitives are very easy to implement, especially since you have already
    been provided with a stable quadratic solver.
    
    There are other possible primitive objects,
    such as superquadrics or tori, although these are harder than
    quadric-based solvers.  For instance, a torus is generated by
    a quartic equation, which can be solved analytically, but
    it's hard to write a numerically stable quartic solver.   Some
    kind of numerical root-solver is often required; reguli-falsi
    is recommended, but isolate the roots first using a 
    geometric approach.
    [Many available references.  Take a look.]
\item[Texture-mapping:] Determine the diffuse reflectance of objects
    based on stochastic or deterministic functions.  
    This requires a function that computes, for each intersection
    point on the surface of a primitive, a set of texture coordinates.
    For instance, for a sphere, you can use the elevation and
    azimuth of the intersection point to index the texture image.
    Alternatively, you can use a projective transformation of
    the MCS coordinates of intersection point.
    Computing the diffuse reflectance procedurally
    as a function of MCS spatial position $(x,y,z)$ can be used to
    generating solid textures, like wood grain or marble. [Perlin,Hart]

    See {\tt http://textures.guinet.com/} for some sample textures.
\item[Bump-mapping:] Involves techniques similar to texture-mapping, but 
the texture functions perturb the object's surface normal, 
rather than diffuse reflectance, at a given point.    
[Blinn]
\item[Lighting Models:] Implement a decent ``physical'' lighting model,
or some other alternative lighting model (like the Blinn-Phong model).
Implement Phong shading (interpolation of vertex normals).
\item[CSG:] Extend the raytracer to provide CSG
operations at model nodes (intersect, union, diff, etc) and extend the
intersection routines to return 1D intersection intervals along the
ray rather than a single intersection distance.   
Then at union nodes merge lists of
intersection intervals passed back from the children, at intersection nodes 
compute the 1D intersection the intervals passed up from the children, etc.
These computations can be performed using a simple count-up-at-entry
and count-down-at-exit algorithm.   
Once the final set of intervals has been computed, take the
first point of the first interval as the intersection point.
\end{description}

\subsection{Improved Efficiency}
The key to improving efficiency is the avoidance of unnecessary work,
or rather, only applying work were it will make a difference.
Some ideas for improving efficiency are:
\begin{description}
% CSK sp2004 -- we now require this...
% \item[Bounding boxes:] Modify your intersection routine to use 
%    hierarchical bounding boxes or spheres to prune useless deep recursion 
%   into the modeling tree.  To handle scaling appropriately, you will 
%  probably need to use SVD.
\item[Intensity thresholds:] Check if the accumulated product of mirror
    reflectances is less than epsilon, then return black without 
    checking for further ray hits (obviously goes along with doing 
    secondary rays for mirror reflection).
\item[Spatial partitioning:] Modify your intersection routine to use
    a space partitioning scheme; e.g. BSP trees, uniform spatial subdivision, 
    or octrees.  Uniform spatial subdivision is particularly
    easy to implement and performs well in practice; a hierarchical
    version can be used as an extension of this in a ray-tracing
    project.
\end{description}

{\bf Note:} simply flattening the DAG and caching transformation 
matrices, as suggested
earlier, is {\em not} enough to get you credit for this objective.

\section{The Interface} 
We have provided a set of stubbed Lua callback functions in C++. 
They implement a superset of the modeling language used in 
Assignment 3; you are expected to extend your code from that assignment.

What we list here are the additional functionality you need to add to the
Assignment 3 language to get full credit in Assignment 4.  You'll 
find that the Assignment 3 source code already provides stubs for this
functionality.

\begin{itemize}
	\item \texttt{gr.nh\_box(\textit{name}, (\textit{x, y, z}), \textit{r})} ---
		Return a non-hierarchical box with name \textit{name}.  
		The box should be aligned with the axes
		of its MCS, with one corner at $(x,y,z)$ and the diagonally
		opposite corner at $(x+r,y+r,z+r)$.
	\item \texttt{gr.nh\_sphere(\textit{name}, (\textit{x, y, z}), \textit{r})} ---
		Create a non-hierarchical sphere with name \textit{name} of
		radius $r$ centered at $(x,y,z)$.
	\item \texttt{gr.cube(\textit{name})} ---
		Return an hierarchical box with name \textit{name}.  
		The box should be aligned with the axes
		of its MCS, with one corner at $(0,0,0)$ and the diagonally
		opposite corner at $(1,1,1)$.
	\item \texttt{gr.mesh(\textit{name}, \{\\
		\hspace*{.5in} \{\textit{v1x, v1y, v1z}\},\\
		\hspace*{.5in} \{\textit{v2x, v2y, v2z}\},\\
		\hspace*{.5in}$\dots$\\
		\hspace*{.5in} \{\textit{vnx, vny, vnz}\},\\
		\}, \{\\
		\hspace*{.5in} \{\textit{p11, p12, p13, ... p1m}\},\\
		\hspace*{.5in} \{\textit{p21, p22, p23, ... p2m}\},\\
		\hspace*{.5in}$\dots$\\
		\hspace*{.5in} \{\textit{pn1, pn2, pn3, ... pnm}\}\\
		\})}

		Create a polygonal mesh named \texttt{name} with the listed 
		vertices and faces.  The first list is a list
		of vertex coordinates, 
		and the second list is a list of polygons.
		Each vertex is given as an $(x,y,z)$ triple, 
		and each polygon is
		a list of integer indices into the vertex list.  
		Vertices are indexed starting at 0.

		It may be assumed that polygons are convex and planar.
		However, polygons may have an arbitrary number of vertices.
	\item \texttt{gr.light(\{\textit{x, y, z}\}, \{\textit{r, g, b}\}, 
		    \{\textit{c0, c1, c2}\})} ---
		Create a point 
		light source at $(x,y,z)$ of intensity $(r,g,b)$.
		The attenuation parameters \textit{c0, c1, c2} 
		specify the attenuation for the particular light source
		according to the formula $1/(c0+c1*r+c2*r^2)$.
        \item \texttt{gr.render(\textit{node, filename, w, h,
                      eye, view, up, fov,
                      ambient, lights})} ---
                  Raytrace an image of $w \times h$ pixels to
                  \textit{filename}. The camera is to be located at position
                  \textit{eye}, looking in direction \textit{view} with
                  \textit{up} pointing up (all of these quantities are
                  three-vectors). A field-of-view of \textit{fov}
                  degrees is to be used. The ambient light should have
                  an intensity of \textit{ambient} (also a
                  three-vector). All lights to be used in raytracing
                  are listed in \textit{lights}.
\end{itemize}

The \texttt{gr.nh\_box} and \texttt{gr.nh\_sphere} callbacks
will allow you to implement a non-hierarchical version of the 
raytracer, since you can place spheres and boxes in arbitrary locations.
Thus you will be able to test aspects of your code such as shading 
and shadows without necessarily having hierarchical transformations
working.  Later, you may find it easier to build scenes  using 
\texttt{gr.sphere} from Assignment~3 and, say, \texttt{gr.cube} for
building a unit cube at the origin.

\section{Donated Code}
The Assignment~4 code is similar to that of Assignment~3. There are a
few new files in the source directory:

\begin{itemize}
  \item \texttt{a4.cpp} -- Contains the \texttt{render()} stub you
    need to implement.
  \item \texttt{image.cpp} -- A class storing rectangular images,
    supporting PNG saving and loading.
  \item \texttt{light.cpp} -- A very simple light class.
  \item \texttt{mesh.cpp} -- A simple mesh class.
  \item \texttt{polyroots.cpp} -- Robust polynomial root solver. You
    may optionally use this in your ray-intersection functions.
\end{itemize}

Furthermore, a number of sample scripts are available in the
\texttt{data/} directory. Some of these require that you run them in
that directory, as they depend on other files found there. A simple
Lua reader for the Alias/Wavefront OBJ mesh format is also provided
(\texttt{readobj.lua}) as well as a sample OBJ file (\texttt{cow.obj}).

You may want to merge in some of the changes you have made to
Assignment~3. For this assignment, however, you do not need to
implement a user interface. Your program will be a command-line
program.

In Assignment~3, \texttt{scene\_lua.cpp} provided a function
\texttt{import\_lua} which returned a scene node after parsing. In
this assignment, a function called \texttt{run\_lua} is provided
instead. It will take care of parsing the scene, and call
\texttt{render()} (from \texttt{a4.cpp}) as necessary. Once
\texttt{run\_lua} is done, the program will finish executing.

\section{Deliverables}
\begin{description}

\item[Executable:] {\tt rt} -- This should take a scene file as its argument.
\item[Sample Images:]
	For a complete assignment, generate three image files 
	using the following Lua scripts (which can be found in 
	{\tt \CourseData/data/A4}):
	\begin{itemize}
		\item {\tt nonhier.lua} -- a non-hierarchical scene
		\item {\tt macho-cows.lua} -- a hierarchical scene with instances
		\item {\tt simple-cows.lua} -- a simplified version of
			{\tt macho-cows.lua}.
	\end{itemize}
	Only one of {\tt simple-cows.lua} and {\tt macho-cows.lua} needs to be
	submitted (although you're welcome to submit both).
	The image files should be in the PNG format and stored 
	in {\tt nonhier.png}, {\tt macho-cows.png} and 
	{\tt simple-cows.png}.
	The scripts {\tt macho-cows.lua} and {\tt simple-cows.lua}
	will fully test all the features
	of your raytracer except your additional feature.  
	If you do not complete the entire assignment, of course, you
        will not be able to render all three scenes.

	In addition, to demonstrate that you've implemented bounding volumes,
	you should make a special renderering of either {\tt nonhier.lua} or
	{\tt macho-cows.lua} where you draw the bounding volumes instead of
	the polygonal meshes.  This image should be stored in
	{\tt nonhier-bb.png} or {\tt macho-cows-bb.png}.

	These image files should be in your {\tt cs488/handin/A4/data} 
	directory.

	You will find sample renderings of all three scenes in the
	image gallery on the course web site.

	Don't forget \texttt{screenshot01.png}.  As usual, it should be
	in your \texttt{cs488/handin/A4} directory.  This should be your
	best image (which should also be in your \texttt{cs488/handin/A4/data} 
	directory, but under a different name).

\item[Sample Script:]
	You need to generate a test script called {\tt sample.lua} and
	render an image from it called {\tt sample.png}.  You should
	place {\tt sample.lua} and all of your images in your \\
	\texttt{cs488/handin/A4/data} directory.
	You do not need to submit a special rendering of your sample
	scene to demonstrate bounding volumes.

	This script or another one should demonstrate your ``additional 
	feature''.   

\item[Additional Documentation:]
	Your {\tt README} needs to contain a description of
		your extra feature and your unique scene(s).
	If you implement an acceleration feature, provide a switch to
	turn it on and off (this can be a compile-time switch: provide
	two executables) and provide comparative timings.  If you
        use external models, please credit where you got them from.
\end{description}

You need to submit at least one image file: {\tt sample.png}.  
This image should have a resolution of at least 500x500.  
You may submit additional images if you wish; 
mention them in your {\tt README}.

If you could not get hierarchical transformations working, submit
an image made without them, and mention that you are missing
hierarchical transformations in your {\tt README}.  You will be
severely penalized if we discover that you misrepresented yourself, 
of course.

There will be a bit of subjective grading of the image created from
the data file you create.  If the image is extremely good, the
instructor may award up to 1 point extra credit.  
If the image is extremely simple, but tests all features,
the TAs may subtract up to one half a mark.  
If the image does not
test all features, more marks may be deducted, since the TAs will
not be able to verify that feature.

\newpage
\section{Objectives: \hfill Assignment 4}
\noindent{\large \bf Due: \AfourDeadline.} \bigskip

\noindent{\large \bf Name: \hrulefill} \bigskip

\noindent{\large \bf UserID: \hrulefill} \bigskip

\noindent{\large \bf Student ID: \hrulefill} \bigskip
\begin{description}
\item[\_\_\_\ {\bf 1:}]
	Objects are visible on the image.  This implies that you can
	generate primary rays, can intersect them with spheres,
	and can generate a PNG output file.
\item[\_\_\_\ {\bf 2}:]
	Cubes and polygonal meshes are properly rendered.
\item[\_\_\_\ {\bf 3}:]
	Objects are correctly ordered from back to front.
\item[\_\_\_\ {\bf 4}:]
	There is a function that generates a background for the
	scene without obscuring the view of any of the objects
	in the scene.  This background is on all the generated
        images.
\item[\_\_\_\ {\bf 5}:]
	Diffuse and specular (Phong) lighting has been accomplished.
\item[\_\_\_\ {\bf 6}:]
	Shadows have been accomplished.
\item[\_\_\_\ {\bf 7}:]
	A script has been supplied that defines and renders a scene 
	different from anyone else's.
\item[\_\_\_\ {\bf 8}:] 
	Hierarchical transformations operate properly.  Spheres and
	cubes can be transformed with affine transformations.
\item[\_\_\_\ {\bf 9}:]
	Bounding volumes (spheres or boxes) have been implemented for
	polygonal objects as demonstrated by a special renderering as
	described above.
\item[\_\_\_\ {\bf 10}:]
	Some extra improvement has been made to the ray tracer.  This may be
	quite small and range from extra efficiency to extra 
	functionality.\footnote{
		State clearly in your {\tt README} what has been done.
		If you implement a feature that improves the performance 
		of the ray tracer, you must state in your manual execution times for 
		your ray tracer running both with and without the feature.
	}
\end{description}

\vspace*{.1in}
{\bf Declaration:}
\begin{quote}
        I have read the statements regarding cheating in the \CourseNumber
        course handouts.
        I affirm with my signature that I have worked out my own solution
        to this assignment, and the code I am handing in is my own.

\end{quote}

        {\bf Signature:}
\vspace*{.1in}



