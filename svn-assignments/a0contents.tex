%\Chapter{Assignment 0}
\begin{quote}
        {\sl 
        ``I want to work on Assignment 0.  I want to do some Tcl
        programming---there's no thinking involved!''
        } \\
        \mbox{}\hfill -- The TA who developed this assignment
\end{quote}

This assignment is due {\bf \AzeroDeadline}.
% , at the beginning of the lecture.
\section{Topics}
\begin{itemize}
\item Familiarise you with the course computing environment.
\item Ensure that your account(s) are set up properly.
\item Provide exposure to gtkmm and a refresher of C++.
\item Provide practice in building/modifying a user interface.
\item Give you a trial run at submitting an assignment.
\end{itemize}

\section{Summary}
This assignment is optional.

If you do not hand it in, you will not lose
marks.  If you do hand it in, it will be marked in the normal way, but the
mark will not be recorded.  
Therefore, if you have any theologically unsound habits that
will cost you marks or if your account is not set up correctly, you will
have a chance to correct things before Assignment~1.   

In past terms, over half the students who did not submit Assignment~0
lost marks on Assignment~1 for setting up their account incorrectly
or for submitting inadequate documentation.  These mistakes could have
been avoided had these students completed Assignment~0.

It is in your best interest to do this assignment.  
You will gain some information and experience which will be
useful for later assignments.  Thus, even if you do not
choose to hand in Assignment~0, you will still eventually need to learn 
what it covers before you can finish later assignments.

Note also that the directions in this assignment are more explicit than
in the remaining assignments.  In the future, the steps to follow will
not be written as explicitly.

\section{Statement}
Do the following:
\begin{enumerate}
    \item
        Set up your account.  To do this, login to a Linux or Unix
        machine in the undergrad environment
        and type ``\texttt{/u/gr/cs488/bin/setup}''.
        Observe that you now have
        the following directory subtree in your home directory:
        \begin{verbatim}
       cs488/
            handin/
                A0/
                    src/
                    data/
                A1/
                    src/
                    data/
                A2/
                    src/
                    data/
                A3/
                    src/
                    data/
                A4/
                    src/
                    data/
                A5/
                    src/
                    data/
        \end{verbatim}
        \item 
            The next steps assume the that \texttt{qmake} and \texttt{qt
            libraries} are installed.

            If you don't have qmake and qt libraries installed, run the following
            script:
            \begin{verbatim}
    sudo apt-get install qt5-qmake qt5-default libqt5opengl5-dev
            \end{verbatim} 
        \item
                You will find several \texttt{.cpp} and \texttt{.hpp}
                files in the \texttt{A0} subdirectory, as well as a
                \texttt{paint488.pro} file. Enter the subdirectory and type 
                \texttt{qmake paint488.pro}. This generates a 
                \texttt{Makefile}. Then, type \texttt{make}. The program 
                should compile and you should be left with an executable called
                \texttt{paint488}.

                The program is a very simple painting program that
                allows you to draw rectangles, ovals and lines.

        \item
                To run the program type \texttt{./paint488} at the
                command line.  To exit you must select
                the ``Quit'' entry under the \texttt{Application} menu.

                Modify \texttt{paintwindow.hpp} and
                \texttt{paintwindow.cpp} so that a ``Quit'' button is
                placed at the bottom of the window.  When this button
                is pressed, the program terminates.

        \item
                Add a Clear entry to the \texttt{Application} menu.  This entry should
                clear the screen.

        \item
                Add a new menu for selecting a colour.  This menu should
                be labeled \texttt{Colour}, and should appear between the
                \texttt{Tools} and \texttt{Help} menus on the menu bar.  You should
                be able to select the following colours: Black, Red,
                Green, Blue.
                Any subsequent draw of a rectangle or oval should fill
                the object with the selected colour.

                The default fill colour should be Black and the
                default shape should be Line.

                If you like, you can add additional fill colours, such
                as Cyan, Magenta, Yellow, Purple, Orange,
                White, but you will only be graded on Black, Red, Green,
                and Blue.
        \item
                Change the Tools menu to use radiobuttons to select
                between the drawing primitives.

        \item
                Add Help entries to Help menu for Rectangles and Ovals.

        \item
                Add the following keyboard shortcuts:
                \begin{description}
                        \item[C:] Clear
                        \item[L:] Line
                        \item[R:] Rectangle
                        \item[O:] Ovals
                \end{description}
		Note that lower case letters should
		trigger the shortcuts.  We have already implemented
		a keyboard shortcut for Quit; you should add your
		keyboard shortcuts to the same handler.

        \item
                Write a Manual and a \texttt{README} for \texttt{paint488} 
                explaining how to run and use it.  
                The Manual should be submitted in hardcopy.
                The \texttt{README} should be in the \texttt{cs488/handin/A0} 
                directory but does not have to be handed in as a hardcopy.  
        \item
                Remove all unnecessary files (eg., 
                        \texttt{*.bak, *{\char126}, core}, etc).

    \item
        From your \texttt{cs488/handin} directory
        run ``\texttt{grsubmit A0}'' to create a checksum
        for your assignment.   Note that \texttt{\CourseData/bin} should
        already be in your search path.
        You are to print out this checksum and
        hand it in with your assignment.

    \item
        Hand in all the things mentioned in the {\bf Assignment Format} 
        documentation that are applicable to this assignment.  Double
        check the objectives to ensure that your assignment meets all of them.
        Make sure you sign the objective sheet and fill in the requested 
        information!

\end{enumerate}

\section{Extras}
If you wish to add some extras (sorry, no bonus points) to the
paint program, consider adding the following:
\begin{itemize}
	\item Rubberbanding.  When drawing an oval or a rectangle,
		it is useful to draw the outline of the shape (or
		even the shape itself) as you move the mouse after
		the mouse button down.
	\item Additional shapes.  Some additional shapes can be
		drawn if you allow multiple mouse presses.
		For example, you could draw a polyline (a sequence
		of line segments), an arbitrary polygon, or B\'ezier
		curves.
	\item Updating/deleting elements.  You might want to change
		the colour of one of the elements you've drawn.  Or
		you might want to delete one of the elements.  This
		would require find the element and either removing
		it from the m\_shape list or modifying its colour.
	\item Save/print.  You could add additional functionality
		to either save the primitives to a file (to be
		read in later) or to print the screen.
\end{itemize}
Note that except for the first one, each of these will likely require
a significant amount of coding.

\section{Donated Code}
In \texttt{\CourseData/data/A0} you will find
\begin{itemize}
        \item \texttt{main.cpp} - The initial startup function
          \texttt{main}.
        \item \texttt{paintcanvas.hpp}, \texttt{paintcanvas.cpp} - The
          canvas onto which the user can draw
        \item \texttt{paintwindow.hpp}, \texttt{paintwindow.cpp} - The
          main window of the application.
        \item \texttt{Makefile} - used to compile your program.
\end{itemize}
These files have been copied to your \texttt{handin/A0} directory.

\section{Deliverables}
\begin{description}
\item[Executables:] \hfill

        \texttt{paint488}.  Be sure to place this in the correct
        directory (under \texttt{A0}).

\item[Source code:] \hfill
        
        All source code should be placed in \texttt{A0/src}. You
        should not have to add any new files, but can if you wish. You
        should not have to change the Makefile. If you wish to add a file 
        to the project, add your files in the \texttt{.pro} file in 
        the \texttt{A0/src} folder.
       
\item[Documentation:] \hfill

        \texttt{README}.  Be sure to place this in the correct directory 
        (under \texttt{A0}).
\end{description}

\newpage
\section{Objectives: \hfill Assignment 0}

\noindent{\large \bf Due: \AzeroDeadline.} \bigskip

\noindent{\large \bf Name: \hrulefill} \bigskip

\noindent{\large \bf UserID: \hrulefill} \bigskip

\noindent{\large \bf Student ID: \hrulefill} \bigskip

\begin{description}
\large
        \item[\_\_\_ 1:]{
            The account is set up correctly.
        }
        \item[\_\_\_ 2:]{
            The checksum program was run, and its output was printed 
            and included in the submitted documentation.
        }
        \item[\_\_\_ 3:]{
            The correct information was handed in as hardcopy, 
                a {\tt README} file exists in \texttt{handin/A0},
                and the executable is in \texttt{handin/A0}.
        }
        \item[\_\_\_ 4:]{
                There is a ``Quit'' button at the bottom
                of the window that terminates the program.
        }
        \item[\_\_\_ 5:]{
                A Clear entry has been added to the Application menu, which
                clears the screen.
        }
        \item[\_\_\_ 6:]{
                A Colour menu with radiobuttons
		has been added with the listed colours.
                After a colour is selected, any subsequent draws of
                a rectangle or oval are filled with that colour,
		and that colour's radiobutton is turned on.
        }
        \item[\_\_\_ 7:]{
                The tools menu has been modified to use radiobuttons
                to select between the graphics primitives.
        }
        \item[\_\_\_ 8:]{
                Keyboard shortcuts have been added for 
                Clear~(C), Lines~(L), Rectangles~(R), and Ovals~(O).
		Both upper and lower case letters should trigger the shortcut.
        }
        \item[\_\_\_ 9:]{
                Help entries for Rectangles and Ovals have been added.
        }
        \item[\_\_\_ 10:]{
            A screenshot of your program is provided in
		{\tt handin/A0/screenshot01.png}.
        }
\end{description}

\input{declare}
